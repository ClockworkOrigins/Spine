\documentclass{article}
\usepackage{color}
\usepackage{accsupp}
\usepackage{hyperref}
\usepackage[ngerman]{babel}
\usepackage[utf8]{inputenc}
\usepackage[T1]{fontenc}
\definecolor{bisque}{rgb}{0.8,0.8,0.92}
\definecolor{green}{rgb}{0.1,0.6,0.2}
\definecolor{brown}{rgb}{0.8,0.4,0.3}

\usepackage{listings}
\lstdefinestyle{basic}
{
  basicstyle=\ttfamily, % tiny,scriptsize,footnotesize,small,normalsize,large,Large,LARGE,huge,Huge
  stringstyle=\color{green}\ttfamily,
  frame=lines, % single none
  captionpos=b,
  columns=fullflexible, % to prevent spaces between characters on copying
  resetmargins=true,
  showspaces=false, % display special space character
  showstringspaces=false, % display special space character in strings
  keepspaces=true, % to copy spaces (not for leading spaces)
  backgroundcolor=\color{bisque}, % background color
  numbers=left, % linenumbers
  numberstyle=\noncopynumber, % line numbers are not copyable
  keywordstyle=\color{blue}, % keyword color
  breaklines=true,
  tabsize=2,
  commentstyle=\color{brown}
}
\newcommand{\noncopynumber}[1]{%
    \BeginAccSupp{method=escape,ActualText={}}%
    #1%
    \EndAccSupp{}%
}




\title{Spine Tutorial 7 \\ Freunde}

\begin{document}

\section{Voraussetzungen}

\begin{enumerate}
\item Ikarus Scriptpaket
\item LeGo Scriptpaket
\item Spine Scriptpaket Version 1.10.0+
\item Tutorial 1 - Initialisierung
\end{enumerate}

\section{Einleitung}

Dieses Tutorial befasst sich mit dem Abfragen der Freunde eines Spielers. Benötigt wird das Modul \textit{SPINE\_MODULE\_FRIENDS}. Nützlich sind Freunde z.B. für gezieltes Matchmaking mit Freunden.

\section{Freunde}

Freunde kann man auf Spine einfach anfragen. Wenn der angefragte die Freundschaft annimmt, wird die Freundschaft auf der Freundeseite in Spine angezeigt und kann ebenfalls über die Scripts abgefragt werden.

\section{Die Funktionen}

Die verschiedenen Funktionen des Freunde-Modul sollen hier kurz vorgestellt werden.

\subsection{Spine\_GetFriendCount}

Mit Spine\_GetFriendCount lässt sich abfragen, wie viele Freunde der User hat. Das ist nötig, um zu wissen, auf wie viele Freunde man zugreifen kann.

\subsection{Spine\_GetFriendName}

Mit Spine\_GetFriendName lässt sich der Names eines Freundes abfragen. Die Freunde sind alphabetisch sortiert und es gibt Spine\_GetFriendCount viele. Angesprochen werden sie über ihren Index, beginnend bei 0, d.h. der höchste Index ist Spine\_GetFriendCount - 1. Das bedeutet, dass man mit

\begin{lstlisting}
Spine_GetFriendName(0);
\end{lstlisting}

den Namen des ersten Freundes abfragen kann.

\end{document}
