\documentclass{article}
\usepackage{color}
\usepackage{accsupp}
\usepackage{hyperref}
\usepackage[ngerman]{babel}
\usepackage[utf8]{inputenc}
\usepackage[T1]{fontenc}
\definecolor{bisque}{rgb}{0.8,0.8,0.92}
\definecolor{green}{rgb}{0.1,0.6,0.2}
\definecolor{brown}{rgb}{0.8,0.4,0.3}

\usepackage{listings}
\lstdefinestyle{basic}
{
  basicstyle=\ttfamily, % tiny,scriptsize,footnotesize,small,normalsize,large,Large,LARGE,huge,Huge
  stringstyle=\color{green}\ttfamily,
  frame=lines, % single none
  captionpos=b,
  columns=fullflexible, % to prevent spaces between characters on copying
  resetmargins=true,
  showspaces=false, % display special space character
  showstringspaces=false, % display special space character in strings
  keepspaces=true, % to copy spaces (not for leading spaces)
  backgroundcolor=\color{bisque}, % background color
  numbers=left, % linenumbers
  numberstyle=\noncopynumber, % line numbers are not copyable
  keywordstyle=\color{blue}, % keyword color
  breaklines=true,
  tabsize=2,
  commentstyle=\color{brown}
}
\newcommand{\noncopynumber}[1]{%
    \BeginAccSupp{method=escape,ActualText={}}%
    #1%
    \EndAccSupp{}%
}




\title{Spine Tutorial 4 \\ Savegame-übergreifende Daten}

\begin{document}

\section{Voraussetzungen}

\begin{enumerate}
\item Ikarus Scriptpaket
\item LeGo Scriptpaket
\item Spine Scriptpaket Version 1.2.0+
\item Tutorial 1 - Initialisierung
\end{enumerate}

\section{Einleitung}

Dieses Tutorial befasst sich mit der Befüllung und Verwaltung von Savegame-übergreifenden Daten. Benötigt wird das Modul \textit{SPINE\_MODULE\_OVERALLSAVE}. Das kann nützlich sein, wenn man beispielsweise Erfolge erstellen möchte nach dem Schema ''Spiel mit allen Gilden einmal abgeschlossen'' oder ''x Gegner besiegt''.

\section{Die Funktionen}

Die verschiedenen Funktionen dieses Moduls sollen hier kurz vorgestellt werden.

\subsection{Spine\_OverallSaveSetString}

Mit Spine\_OverallSaveSetString lässt sich ein beliebiger String-Wert speichern. Dazu wird ein Schlüsselwert zum Identifizieren benötigt. Ein Beispiel:

\begin{lstlisting}
Spine_OverallSaveSetString("LastTalkedNpc", "Xardas");
\end{lstlisting}

\subsection{Spine\_OverallSaveGetString}

Mit Spine\_OverallSaveSetString lässt sich ein gespeicherter Wert aus dem Save wieder auslesen. Das bedeutet, dass man mit

\begin{lstlisting}
Spine_OverallSaveGetString("LastTalkedNpc");
\end{lstlisting}

den Wert "Xardas" bekommt, wenn man das Beispiel oben ausgeführt hat.

\subsection{Spine\_OverallSaveSetInt}

Mit Spine\_OverallSaveSetInt lässt sich ein Integer-Wert, also eine Zahl, speichern. Das funktioniert analog zu den Strings. Das bedeutet, dass man mit

\begin{lstlisting}
Spine_OverallSaveSetInt("MonstersKilled", 10);
\end{lstlisting}

die Anzahl an getöteten Monstern speichern kann.

\subsection{Spine\_OverallSaveGetInt}

Mit Spine\_OverallSaveGetInt lässt sich der Wert dann auch genauso einfach wieder auslesen. Das bedeutet, dass man mit

\begin{lstlisting}
Spine_OverallSaveGetInt("MonstersKilled");
\end{lstlisting}

den Wert 10 bekommt, wenn man das Beispiel oben ausgeführt hat.

\end{document}
